\documentclass{article}
\usepackage{fancyhdr} % for pretty formatting
\usepackage{amsmath} % for matrices
\usepackage{amssymb} % for bold text
\usepackage{pgfplots} % for graphs
\usepackage{hyperref} % for hyperlinks
\pgfplotsset{compat=1.18}

\usepackage{lipsum} % For dummy text
\usepackage{cite} % For citations

\pagestyle{fancy}
\fancyhf{} % Clear all header and footer fields

\lhead{Joshua Dunne}
\rhead{\thepage} % Displays the current page number
\lfoot{MATH620}
\cfoot{Homework 2}

\begin{document}
    \section{Question 1}
        \subsection{Givens}
            \paragraph{Modes of transport}
                $\mathbf{v}_1=\begin{bmatrix}1\\1\\1\end{bmatrix}$
                $\mathbf{v}_2=\begin{bmatrix}6\\3\\8\end{bmatrix}$
                $\mathbf{v}_3=\begin{bmatrix}4\\1\\6\end{bmatrix}$
        \subsection{Questions}
            \renewcommand{\labelenumi}{\alph{enumi}}
            \begin{enumerate}
                \item
                    \begin{description}
                        \item[Question:]Is there anywhere in $\mathbb{R}^3$ that old man Gauss can hide?
                        \item[Answer:]
                            Maybe, let's do a quick RREF and have a look
                            \[
                            \begin{bmatrix}
                            \begin{array}{ccc|c}
                                1 & 6 &  4 & x \\
                                0 & 3 &  8 & y \\
                                4 & 1 &  6 & z
                            \end{array}
                            \end{bmatrix}
                            \rightarrow
                            \begin{bmatrix}
                            \begin{array}{ccc|c}
                                1 & 0 & -2 & 0 \\
                                0 & 1 &  1 & 0 \\
                                0 & 0 &  0 & 1
                            \end{array}
                            \end{bmatrix}
                            \]
                            As we're short a pivot we can be sure of a couple things.
                            The third row is missing both a pivot, but, as importantly,
                            it is missing a value in the agumented column.
                            Given these things together, we can be sure that, one of the vectors is 
                            linearly dependent on the others, and that there are infinite solutions
                    \end{description}
            \end{enumerate}
            \section{Question 2}
                \subsection{Givens}
                    \paragraph{Suppose}
                        \[
                        \mathbf{v_1}=\begin{bmatrix}1\\-1\\2\end{bmatrix}
                        \mathbf{v_2}=\begin{bmatrix}-1\\2\\3\end{bmatrix}
                        \mathbf{v_3}=\begin{bmatrix}2\\3\\h\end{bmatrix}
                        \]
                \subsection{Question}
                    \begin{description}
                        \item[Question:]
                            For what values of $h$ are the vectors 
                            $\mathbf{v_1}$, $\mathbf{v_2}$, and $\mathbf{v_3}$ 
                            linearly dependent?
                        \item[Answer:]
                            We can do this with the determinant this time around. The system
                            should result in a determinant that is $0$.
                            \[
                            det(
                                \begin{bmatrix}
                                    1 & -1 & 2 \\
                                    -1 & 2 & 3 \\
                                    2 & 3 & h
                                \end{bmatrix}
                            )
                            =0
                            \]
                            \[
                            1(2h-9)+1(-h-6)+2(-3-4)=0
                            \]
                            \[
                            2h-9-h-6-6-8=0
                            \]
                            \[
                            h-29=0
                            \]
                            \[
                            h=29
                            \]
                        \item[Check:]
                            So, given a value of $h=29$ we can check our work by calculating
                            \[
                            \mathbf{v_1}=\begin{bmatrix}1\\-1\\2\end{bmatrix}
                            \mathbf{v_2}=\begin{bmatrix}-1\\2\\3\end{bmatrix}
                            \mathbf{v_3}=\begin{bmatrix}2\\3\\29\end{bmatrix}
                            \]
                            \[
                            \begin{bmatrix}
                            \begin{array}{ccc|c}
                                1  & -1 & 2 & x \\
                                -1 & 2  & 3 & y \\
                                2  & 3  & 29 & z
                            \end{array}
                            \end{bmatrix}
                            \rightarrow
                            \begin{bmatrix}
                            \begin{array}{ccc|c}
                                1 & 0 & 7 & 0 \\
                                0 & 1 & 1 & 0 \\
                                0 & 0 & 0 & 1
                            \end{array}
                            \end{bmatrix}
                            \]
                            As we have a row of all zeros, we can be sure that the vectors are linearly dependent.

                    \end{description}
\end{document}