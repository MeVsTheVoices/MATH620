\documentclass{article}
\usepackage{fancyhdr} % for pretty formatting
\usepackage{amsmath} % for matrices
\usepackage{amssymb} % for bold text
\usepackage{pgfplots} % for graphs
\usepackage{hyperref} % for hyperlinks
\pgfplotsset{compat=1.18}

\usepackage{lipsum} % For dummy text
\usepackage{cite} % For citations

\pagestyle{fancy}
\fancyhf{} % Clear all header and footer fields

\lhead{Joshua Dunne}
\rhead{\thepage} % Displays the current page number
\lfoot{MATH620}
\rfoot{Unit 2}
\cfoot{Homework 3}

\begin{document}
    \section{Question 1}
        \paragraph{Given} 
            \begin{center}
                Given
                $\vec{v_3} = \begin{bmatrix}1\\2\\3\\4\end{bmatrix}$
                is 
                $\vec{v_3} \in \text{span}\{\vec{v_1}, \vec{v_2}\}$?
            \end{center}
            \begin{center}
                where
                $\vec{v_1} = \begin{bmatrix}1\\1\\-1\\1\end{bmatrix}$ and
                $\vec{v_2} = \begin{bmatrix}1\\-1\\1\\0\end{bmatrix}$.
            \end{center}
        \paragraph{Solution}
           We're given thast we can consider these vectors as
            \[
                A = \begin{bmatrix}1 & 1\\-1 & 1\end{bmatrix},
                B = \begin{bmatrix}1 & -1\\1 & 0\end{bmatrix},
                C = \begin{bmatrix}1 & 2\\3 & 4\end{bmatrix}
            \]
            We're still asking the same question,
            so let's see how that breaks down
            \[
                c_1\begin{bmatrix}1 & 1\\-1 & 1\end{bmatrix} +
                c_2\begin{bmatrix}1 & -1\\1 & 0\end{bmatrix} =
                \begin{bmatrix}1 & 2\\3 & 4\end{bmatrix}
            \]
            This gives us the following system of equations
            \begin{align*}
                c_1 + c_2 &= 1\\
                c_1 - c_2 &= 2\\
                c_1 &= 3\\
                c_1 &= 4
            \end{align*}

            Seen as we can't reconcile the last two equations,
            we can see that this system is inconsistent.
    \section{Question 2}
        \paragraph{Restate}
            Is $\mathbf{v}_4=\begin{bmatrix}1\\1\\1\end{bmatrix}$
            $\in \text{span}\{\mathbf{v}_1, \mathbf{v}_2, \mathbf{v}_3\}$?
            \subparagraph{Values}
                \[
                    \mathbf{v}_1=\begin{bmatrix} 1\\-2\\ 0\end{bmatrix},
                    \mathbf{v}_2=\begin{bmatrix} 0\\ 1\\-1\end{bmatrix},
                    \mathbf{v}_3=\begin{bmatrix}-2\\ -3\\1\end{bmatrix}
                \]
            \subparagraph{Form}
                $\text{span}\{p(x),q(x),r(x)\}$
                is of the form
                \begin{align*}
                    p(x) &= 1 - 2x + 0x^2\\
                    q(x) &= 0 + 1x - 1x^2\\
                    r(x) &= -2 - 3x + 1x^2
                \end{align*}
        \paragraph{Investigation}
\end{document}