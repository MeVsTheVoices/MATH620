\documentclass{article}
\usepackage{fancyhdr} % for pretty formatting
\usepackage{amsmath} % for matrices
\usepackage{amssymb} % for bold text
\usepackage{pgfplots} % for graphs
\usepackage{hyperref} % for hyperlinks
\pgfplotsset{compat=1.18}

\usepackage{lipsum} % For dummy text
\usepackage{cite} % For citations

\pagestyle{fancy}
\fancyhf{} % Clear all header and footer fields

\lhead{Joshua Dunne}
\rhead{\thepage} % Displays the current page number
\lfoot{MATH620}
\rfoot{Unit 2}
\cfoot{Homework 3}

\begin{document}
    \section{Question 1}
        \paragraph{Given} 
            \begin{center}
                Given
                $\vec{v_3} = \begin{bmatrix}1\\2\\3\\4\end{bmatrix}$
                is 
                $\vec{v_3} \in \text{span}\{\vec{v_1}, \vec{v_2}\}$?
            \end{center}
            \begin{center}
                where
                $\vec{v_1} = \begin{bmatrix}1\\1\\-1\\1\end{bmatrix}$ and
                $\vec{v_2} = \begin{bmatrix}1\\-1\\1\\0\end{bmatrix}$.
            \end{center}
        \paragraph{Solution}
           We're given thast we can consider these vectors as
            \[
                A = \begin{bmatrix}1 & 1\\-1 & 1\end{bmatrix},
                B = \begin{bmatrix}1 & -1\\1 & 0\end{bmatrix},
                C = \begin{bmatrix}1 & 2\\3 & 4\end{bmatrix}
            \]
            We're still asking the same question,
            so let's see how that breaks down
            \[
                c_1\begin{bmatrix}1 & 1\\-1 & 1\end{bmatrix} +
                c_2\begin{bmatrix}1 & -1\\1 & 0\end{bmatrix} =
                \begin{bmatrix}1 & 2\\3 & 4\end{bmatrix}
            \]
            This gives us the following system of equations
            \begin{align*}
                c_1 + c_2 &= 1\\
                c_1 - c_2 &= 2\\
                c_1 &= 3\\
                c_1 &= 4
            \end{align*}

            Seen as we can't reconcile the last two equations,
            we can see that this system is inconsistent.
    \section{Question 2}
        \paragraph{Restate}
            Is $\mathbf{v}_4=\begin{bmatrix}1\\1\\1\end{bmatrix}$
            $\in \text{span}\{\mathbf{v}_1, \mathbf{v}_2, \mathbf{v}_3\}$?
            \subparagraph{Values}
                \[
                    \mathbf{v}_1=\begin{bmatrix} 1\\-2\\ 0\end{bmatrix},
                    \mathbf{v}_2=\begin{bmatrix} 0\\ 1\\-1\end{bmatrix},
                    \mathbf{v}_3=\begin{bmatrix}-2\\ -3\\1\end{bmatrix}
                \]
            \subparagraph{Form}
                $\text{span}\{p(x),q(x),r(x)\}$
                is of the form
                \begin{align*}
                    p(x) &= 1 - 2x + 0x^2\\
                    q(x) &= 0 + 1x - 1x^2\\
                    r(x) &= -2 - 3x + 1x^2
                \end{align*}
        \paragraph{Investigation}
            \begin{align*}
                1 &= 1c_1 + 0c_2 -2c_3\\
                1 &= -2c_1 + 1c_2 -3c_3\\
                1 &= 0c_1 -1c_2 + 1c_3 
            \end{align*}
            \subparagraph{Augment}
                We can express this as an augmented matrix
                \[
                \begin{bmatrix}
                \begin{array}{ccc|c}
                    1 &  0 &  -2 & 1 \\
                    -2 &  1 &  -3 & 1 \\
                    0 & -1 &   1 & 1
                \end{array}
                \end{bmatrix}
                \rightarrow
                \begin{bmatrix}
                \begin{array}{ccc|c}
                    1 &  0 &  -2 & 1 \\
                    0 &  1 &  0 & -1/2 \\
                    0 &  0 &  1 & 5/2
                \end{array}
                \end{bmatrix}
                \]
            \subparagraph{Analysis}
                Not only consistent, but unique as well!
                So. Yes. We can use these coefficients
                to express $\mathbf{v}_4$ as a linear combination
                of $\mathbf{v}_1, \mathbf{v}_2, \mathbf{v}_3$.
    \section{Question 3}
        \subsection{Question}
            Using RREF. How would you determine whether a system of linear equations
            is consistent, inconsistent, or has infinite solutions?
        \subsection{Answer}
            Been looking forwards to this one. Was one of the first things I had to spend
            a moment clarifying for myself.
            \paragraph{Inconsistent}
                This one's the easiest.
                If the system that I produce in anyway leads to a contradiction,
                then the system is inconsistent. Too. If there is no solution
                to a row in the RREF form, then the system is inconsistent.
                \subparagraph{Example}
                    \[
                    \begin{bmatrix}
                    \begin{array}{ccc|c}
                        1 &  0 &  -2 & 1 \\
                        0 &  1 &  0 & -1/2 \\
                        0 &  0 &  0 & 5
                    \end{array}
                    \end{bmatrix}
                    \]
                    Here we're lead to the conclusion that $0=5$,
                    which is clearly a contradiction.
                \subparagraph{Heuristic}
                    If you have a row of all zeros on the left side
                    and a non-zero value on the right side, then
                    the system is inconsistent.
            \paragraph{Consistent}
                We can just be lazy and say that consistency is the absense
                of a contradiction. We can subtype this into two categories,
                unique solutions and infinite solutions.
                \subparagraph{Infinite}
                    \[
                    \begin{bmatrix}
                    \begin{array}{ccc|c}
                        1 &  0 &  -2 & 1 \\
                        0 &  1 &  0 & -1/2 \\
                        0 &  0 &  0 & 0
                    \end{array}
                    \end{bmatrix}
                    \]
                    Here we can see that there is a free variable,
                    which means that there are infinite solutions.
                    We could express the solution set as
                    \[
                        \begin{bmatrix}
                            1 + 2t\\
                            -1/2\\
                            t
                        \end{bmatrix}
                        \text{ for any } t \in \mathbb{R}
                    \]
                    Hence, infinite solutions.
                \subparagraph{Unique}
                    \[
                    \begin{bmatrix}
                    \begin{array}{ccc|c}
                        1 &  0 &  -2 & 1 \\
                        0 &  1 &  0 & -1/2 \\
                        0 &  0 &  1 & 5/2
                    \end{array}
                    \end{bmatrix}
                    \]
                    Here we can see that there is a unique solution.
                    Way more boring.
                    We can express the solution set as
                    \[
                        \begin{bmatrix}
                            1\\
                            -1/2\\
                            5/2
                        \end{bmatrix}
                    \]
                    Hence, a unique solution.
\end{document}