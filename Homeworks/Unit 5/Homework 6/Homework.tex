\documentclass{article}
\usepackage{fancyhdr} % for pretty formatting
\usepackage{amsmath} % for matrices
\usepackage{amssymb} % for bold text
\usepackage{pgfplots} % for graphs
\usepackage{hyperref} % for hyperlinks
\pgfplotsset{compat=1.18}
\usepackage{enumitem} % for custom lists

\usepackage{tikz}
\usepackage[svgnames]{xcolor}

\usepackage{lipsum} % For dummy text
\usepackage{cite} % For citations

\pagestyle{fancy}  
\fancyhf{} % Clear all header and footer fields

\usepackage{tcolorbox} % Required for tcolorbox

\newtcolorbox{solutioncheck}{
    colback=green!10, % Background color
    colframe=gray!50, % Frame color
    boxsep=5pt, % Padding
    arc=4pt, % Rounded corners
    title=Checking solution, % Optional title for the aside
    fonttitle=\bfseries, % Title font style
} % for Asides

\lhead{Joshua Dunne}
\rhead{\thepage} % Displays the current page number 
\lfoot{MATH620}
\rfoot{Unit 4}
\cfoot{Homework 5}

\begin{document}
    \section{Question 1}
        \paragraph{Determine}
            Examine the transformations below and Determine
            whether they are linear. Justify this by the
            definition as given in class, showing that it does
            indeed hold, or showing where on which condition things
            break down.
        \subsection{Part a}
            $T: \mathbb{R}^n\rightarrow \mathbb{R}^n$
            by
            $T(\mathbf{x})=a\mathbf{x}+\mathbf{b}\quad \forall{b}\in{\mathbf{R}^{n\times n}}$
            \paragraph{Checking homogeneity}
                \subparagraph{Left hand side}
                    $T(c\mathbf{x})=\mathbf{a} (c \mathbf{x})+\mathbf{b}=c(\mathbf{ax})+\mathbf{b}$
                \subparagraph{Right hand side}
                    $c T(\mathbf{x})=c(\mathbf{ax}+\mathbf{b})=c(\mathbf{ax})+c\mathbf{b}$
                \subparagraph{Conclusion}
                    Homogeneity breaks as the two sides are not equivalent.
        \subsection{Part B}
            $T: \mathbb{R}^{n \times n} \rightarrow \mathbb{R}^{n \times n}$
            by
            $T(\mathbf{x})=\mathbf{Ax}-\mathbf{xA}$
            \paragraph{Checking homogeneity}
                \subparagraph{Left hand side}
                    $cT(\mathbf{x})=c(\mathbf{Ax}-\mathbf{xA})=c\mathbf{Ax}-c\mathbf{xA}$
                \subparagraph{Right hand side}
                    $T(c\mathbf{x})=\mathbf{A}(c\mathbf{x})-c\mathbf{xA}$
                \subparagraph{Conclusion}
                    Homogeneity checks out. We can bubble the c outwards and equate the two sides.
            \paragraph{Checking additivity}
                \subparagraph{Left hand side}
                    $T(\mathbf{u}+\mathbf{v})=\mathbf{A}(\mathbf{u}+\mathbf{v})-(\mathbf{u}+\mathbf{v})\mathbf{A}=\mathbf{Au}+\mathbf{Av}-\mathbf{uA}-\mathbf{vA}$
                \subparagraph{Right hand side}
                    $T(\mathbf{u})+T(\mathbf{v})=(\mathbf{Au}-\mathbf{uA})+(\mathbf{Av}-\mathbf{vA})=\mathbf{Au}+\mathbf{Av}-\mathbf{uA}-\mathbf{vA}$
                \subparagraph{Conclusion}
                    Unlike above, additivity checks out. I've seen some other places have more requirements,
                    such as $\langle 0, 0, \dots \rangle \rightarrow \langle 0, 0, \dots \rangle$
                    but all of those follow from homogeneity and additivity.

\end{document}