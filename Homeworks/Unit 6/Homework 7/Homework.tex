\documentclass{article}
\usepackage{fancyhdr} % for pretty formatting
\usepackage{amsmath} % for matrices
\usepackage{amssymb} % for bold text
\usepackage{pgfplots} % for graphs
\usepackage{hyperref} % for hyperlinks
\pgfplotsset{compat=1.18}
\usepackage{enumitem} % for custom lists

\usepackage{tikz}
\usepackage[svgnames]{xcolor}
\usetikzlibrary {positioning}
\definecolor {processblue}{cmyk}{0.96,0,0,0}

\usepackage[math]{cellspace}
\setlength\cellspacetoplimit{3pt}
\setlength\cellspacebottomlimit{3pt} % needed for fractional matrices

\usepackage{lipsum} % For dummy text
\usepackage{cite} % For citations

\pagestyle{fancy}  
\fancyhf{} % Clear all header and footer fields

\usepackage{tcolorbox} % Required for tcolorbox

\newtcolorbox{solutioncheck}{
    colback=green!10, % Background color
    colframe=gray!50, % Frame color
    boxsep=5pt, % Padding
    arc=4pt, % Rounded corners
    title=Checking solution, % Optional title for the aside
    fonttitle=\bfseries, % Title font style
} % for Asides

\lhead{Joshua Dunne}
\rhead{\thepage} % Displays the current page number 
\lfoot{MATH620}
\rfoot{Unit 4}
\cfoot{Homework 5}

\begin{document}
    \section{Question 1}
        \paragraph{Given}
            We're given a set of points that have different offsets
            based on two different basis vectors. First, let's grab
            these points and where they map to in the new basis.
        \subsection{Part A}
            \begin{center}
                \Large
            \begin{tabular}{c c c c}
                $\mathbf{B}_x$ & $\mathbf{B}_y$ & $\mathbf{B}'_x$ & $\mathbf{B}'_y$ \\
                \hline
                1 & 4 & 0 & 1\\
                4 & 7 & 1 & 2\\
                -8 & -5 & -3 & -2\\
                7 & 1 & 1 & 0\\
                $5\frac{1}{2}$ & -5 & 3 & $-\frac{1}{2}$\\
                -6 & $-1\frac{1}{2}$ & $-2\frac{1}{2}$ & -1
            \end{tabular}
            \end{center}
            Where I've denoted $\mathbf{B}$ as the black basis, and $\mathbf{B}'$ as the red.
            These taken together give us plenty of information to find a change of basis.
        \subsection{Part B}
            The points above are of the form $[\mathbf{v}_n]_{\mathbf{B}}\, \rightarrow \,[\mathbf{v}_n]_{\mathbf{B}'}\, \forall{n}$.
            So we can form a system mapping two columns of the first to that of the second.
            \paragraph{The system}
                Let $\mathbf{P}$ be the change of basis we're attempting to find.
                \[
                    \mathbf{P}\begin{bmatrix}1 & 4\\4 & 7\end{bmatrix}
                    =
                    \begin{bmatrix}0 & 1\\1 & 2\end{bmatrix}
                \]
                We need to clear out the left hand side, so take the matrix besides $\mathbf{P}$
                as $\mathbf{A}$, and find the inverse. As $\mathbf{A}\mathbf{A}^{-1}=\mathbf{I}_2$
                \[
                    \mathbf{P}=
                    \begin{bmatrix}0 & 1\\1 & 2\end{bmatrix}
                    \begin{bmatrix}1 & 4\\4 & 7\end{bmatrix}^{-1}
                    =
                    \begin{bmatrix}0 & 1\\1 & 2\end{bmatrix}
                    \frac{1}{9}
                    \begin{bmatrix}7 & -4\\-4 & 1\end{bmatrix}
                    =
                    \frac{1}{9}
                    \begin{bmatrix}4 & -1\\1 & 2\end{bmatrix}
                \]
                \begin{solutioncheck}
                    Let's check this solution by multiplying $\mathbf{P}$ by another point
                    in the original basis, and see if it maps to the correct point in the new basis.
                    \[
                        \mathbf{P}\begin{bmatrix}-8\\-5\end{bmatrix}
                        =
                        \frac{1}{9}
                        \begin{bmatrix}4 & -1\\1 & 2\end{bmatrix}
                        \begin{bmatrix}-8\\-5\end{bmatrix}
                        =
                        \frac{1}{9}
                        \begin{bmatrix}-32 + 5\\-8 - 10\end{bmatrix}
                        =
                        \frac{1}{9}
                        \begin{bmatrix}-27\\-18\end{bmatrix}
                        =
                        \begin{bmatrix}-3\\-2\end{bmatrix}
                    \]
                    This matches the expected output, so our change of basis is correct.
                \end{solutioncheck}
        \subsection{Part C}
            Now we're after a way to go in reverse. We could be super diligent
            and do the same process, but we can also just take the inverse of
            $\mathbf{P}$.
            \[
                \mathbf{P}^{-1}
                =
                \begin{bmatrix}
                    \frac{4}{9} & -\frac{1}{9} \\
                    \frac{1}{9} & \frac{2}{9}
                \end{bmatrix}
            \]
            \begin{solutioncheck}
                Let's check this solution by multiplying $\mathbf{P}^{-1}$ by
                another point in the red basis, and see if it maps to the correct point in the black basis.
                \[
                    \mathbf{P}^{-1}[\mathbf{v}_2]_{\mathbf{B}'}
                    =
                    \begin{bmatrix}
                        2 & 1\\
                        -1 & 4
                    \end{bmatrix}
                    \begin{bmatrix}
                        1\\2
                    \end{bmatrix}
                    =
                    \begin{bmatrix}
                        4\\7
                    \end{bmatrix}
                    =
                    [\mathbf{v}_2]_{\mathbf{B}}
                \]
                Again, seems the intuition is correct. For this point atleast,
                it seems our change of basis is mapping back from black to red.
            \end{solutioncheck}
        \subsection{Part D}
            Ok. I'm reading stretch so, eigenvalues, let's give things a name before we get going
            \paragraph{Given}
                \subparagraph{First line}
                    We are given a stretch factor of $2$ corresponding to the line $y=-\frac{1}{2}x$
                    So that gives us
                    \[
                        T(\begin{bmatrix}2\\-1\end{bmatrix})=\begin{bmatrix}4\\-2\end{bmatrix}
                    \]
                    Where we take $\lambda_1=2$ as the eigenvalue
                \subparagraph{Second line}
                    For the second we're given a stretch factor of $-1$ and a line $y=4x$. From this
                    \[
                        T(\begin{bmatrix}1\\4\end{bmatrix})=\begin{bmatrix}-1\\-4\end{bmatrix}
                    \]
                    Where we take $\lambda_2=-1$ as the eigenvalue
            \paragraph{Forming a system}
                We know that we want a $2x2$ matrix, let's call it $\mathbf{F}$, such that
                \[
                    \mathbf{F}=\begin{bmatrix}a & b\\c & d\end{bmatrix}
                \]
                where
                \begin{center}
                \begin{minipage}{0.49\textwidth}
                    \centering
                    \[
                        \mathbf{F}\begin{bmatrix}2\\-1\end{bmatrix}=\begin{bmatrix}4\\-2\end{bmatrix}
                    \]
                \end{minipage}\hfill
                \begin{minipage}{0.49\textwidth}
                    \centering
                    \[ \mathbf{F}\begin{bmatrix}1\\4\end{bmatrix}=\begin{bmatrix}-1\\-4\end{bmatrix} \]
                \end{minipage}
                \end{center}
                solving for $a$, $b$, $c$, and $d$ we get
                \begin{align*}
                    2a-b&=4\\
                    2c-d&=-2\\
                    a+4b&=-1\\
                    c+4d&=-4
                \end{align*}
                which results $a=\frac{5}{3}$, $b=-\frac{2}{3}$, $c=-\frac{4}{3}$, and $d=-\frac{2}{3}$.
                Giving us a final matrix of
                \[
                    \mathbf{F}=\begin{bmatrix}\frac{5}{3} & -\frac{2}{3} \\ \frac{4}{3} & -\frac{2}{3}\end{bmatrix}
                    =
                    \frac{1}{3}\begin{bmatrix}5 & -2 \\ 4 & -2\end{bmatrix}
                \]
            \paragraph{Mapping red to black}
                \subparagraph{Given}
                    \[
                        [\mathbf{x}]_{\mathbf{r}}
                        =
                        \begin{bmatrix}
                            1\\\frac{1}{2}    
                        \end{bmatrix}
                    \]
                \subparagraph{Work}
                    We need to first map this back to the standard basis, then apply our stretch,
                    then map it back to the red basis. We can borrow our $\mathbf{P}$ from before
                    \[
                        \mathbf{P}[\mathbf{x}]_{\mathbf{r}}=
                        \begin{bmatrix}
                            \frac{4}{9} & -\frac{1}{9} \\
                            \frac{1}{9} & \frac{2}{9}
                        \end{bmatrix}
                        \begin{bmatrix}
                            1\\\frac{1}{2}
                        \end{bmatrix}
                        =
                        \begin{bmatrix}
                            \frac{7}{18}\\\frac{2}{9}
                        \end{bmatrix}
                        =
                        [\mathbf{x}]_{\mathbf{e}}
                    \]
                    And applying our $\mathbf{F}$ stretch
                    \[
                        \mathbf{F}[\mathbf{x}]_{\mathbf{e}}=
                        \frac{1}{3}\begin{bmatrix}5 & -2 \\ 4 & -2\end{bmatrix}
                        \begin{bmatrix}
                            \frac{7}{18}\\\frac{2}{9}
                        \end{bmatrix}
                        =
                        \begin{bmatrix}
                            \frac{1}{2}\\ \frac{10}{27}
                        \end{bmatrix}
                    \]
                    The answer above is in the black basis.
            \paragraph{Mapping black to red}
                \subparagraph{Given}
                    \[
                        [\mathbf{x}]_{\mathbf{e}}
                        =
                        \begin{bmatrix}
                            -3\\3
                        \end{bmatrix}
                    \]
                    We want to apply our transformation, then represent our
                    answer in the red basis.
                \subparagraph{Work}
                    Applying our $\mathbf{F}$ stretch
                    \[
                        \mathbf{F}[\mathbf{x}]_{\mathbf{e}}=
                        \frac{1}{3}\begin{bmatrix}5 & -2 \\ 4 & -2\end{bmatrix}
                        \begin{bmatrix}
                            -3\\3
                        \end{bmatrix}
                        =
                        \begin{bmatrix}
                            -7\\-6
                        \end{bmatrix}
                    \]
                    Now, back to the red basis, we borrow $P^{-1}$ from above
                    \[
                        \mathbf{P}^{-1}\mathbf{F}[\mathbf{x}]_{\mathbf{e}}=
                        \begin{bmatrix}
                            \frac{4}{9} & -\frac{1}{9} \\
                            \frac{1}{9} & \frac{2}{9}
                        \end{bmatrix}
                        \begin{bmatrix}
                            -7\\-6
                        \end{bmatrix}
                        =
                        \begin{bmatrix}
                            -\frac{22}{9}\\-\frac{19}{9}
                        \end{bmatrix}
                    \]
    \section{Question 2}
        Our boy Gauss is back, making our lives difficult again. We're keeping the modes of
        transport we had originally as
        \[
            \beta
            =\left\{
                \begin{bmatrix}
                    3\\1
                \end{bmatrix}
                \begin{bmatrix}
                    1\\2
                \end{bmatrix}
            \right\}
        \]
        And, as I would've guessed\dots changing basis.
        \subsection{Part A}
            \paragraph{Given}
                For one old Uncle Cramer, we're given that in the standard basis
                his house is located at 
                $\left[\mathbf{w}_{\mathbf{e}}\right]=\begin{bmatrix}25\\71\end{bmatrix}$
            \paragraph{Changing basis}
                I want to clarify my understanding for a moment. Above is simply
                analogous to saying that
                \[
                    \begin{bmatrix}1 & 0\\0 & 1\end{bmatrix}
                    \begin{bmatrix}25\\71\end{bmatrix}
                    =
                    \begin{bmatrix}25\\71\end{bmatrix}
                \]
                And our desire to change basis forms a system like so
                \[
                    \begin{bmatrix}3 & 1 \\ 1 & 2\end{bmatrix}
                    \begin{bmatrix}c_1\\c_2\end{bmatrix}
                    =
                    \begin{bmatrix}25\\71\end{bmatrix}
                \]
                Let $\mathbf{P}$ be formed from the columns of our transport basis.
                It's inverse is then
                \[
                    \mathbf{P}^{-1}
                    =
                    \begin{bmatrix}3 & 1 \\ 1 & 2\end{bmatrix}^{-1}
                    =
                    \frac{1}{5}
                    \begin{bmatrix}2 & -1 \\ -1 & 3\end{bmatrix}
                \]
                Lastly, applying our initial vector we get
                \[
                    \mathbf{P}^{-1}
                    \left[\mathbf{w}\right]_{\mathbf{e}}
                    =
                    \left[\mathbf{w}\right]_{\beta}
                    =
                    \frac{1}{5}
                    \begin{bmatrix}2 & -1 \\ -1 & 3\end{bmatrix}
                    \begin{bmatrix}25\\71\end{bmatrix}
                    =
                    \frac{1}{5}
                    \begin{bmatrix}-21\\188\end{bmatrix}
                    =
                    \begin{bmatrix}-\frac{21}{5}\\\frac{188}{5}\end{bmatrix}
                \]
                So, we'd travel $-\frac{21}{5}$ times by hoverboard, and $\frac{188}{5}$ times by magic carpet.
                Not exactly satisfying, but, hey.
        \subsection{Part B}
            \paragraph{Given}
                Now in reverse we're given the location of a museum expressed
                in the transport basis as $\left[\mathbf{v}\right]_{\beta}=\begin{bmatrix}8\\3\end{bmatrix}$
            \paragraph{Changing basis}
                At the risk of turning one mistake in to two, this should be simply the inverse, ie. the original basis
                times the vector in the transport basis.
                \[
                    \mathbf{P}\left[\mathbf{v}\right]_{\beta}
                    =
                    \left[\mathbf{v}\right]_{\mathbf{e}}
                    =
                    \begin{bmatrix}
                        3 & 1 \\ 1 & 2
                    \end{bmatrix}
                    \begin{bmatrix}
                        8\\3
                    \end{bmatrix}
                    =
                    \begin{bmatrix}
                        27\\14
                    \end{bmatrix}
                \]
    \section{Question 3}
        \paragraph{Given}
            \[
                \mathbf{A}=
                \begin{bmatrix}-3&4\\-2&6\end{bmatrix}
            \]
        \paragraph{Running the characteristic}
            Gonna just do the work here, then answer the questions as they come along
            we're gonna need the characteristic polynomial for the first two questions anyways
        \subsection{Characteristic equation}
            \[
                \text{det}(\mathbf{A}-\lambda\mathbf{I}_2)=0
                =
                \begin{bmatrix}-3 & 4\\-2 & 6\end{bmatrix}
                -
                \begin{bmatrix}\lambda & 0\\0 & \lambda\end{bmatrix}
                =
                \begin{bmatrix}-3-\lambda & 4\\-2 & 6-\lambda\end{bmatrix}
            \]
            \[
                \text{det}(\mathbf{A}-\lambda\mathbf{I}_2)=0
                =
                (-3-\lambda)(6-\lambda)+8
                =
                -\lambda^2-3\lambda-10
            \]
            So, take $\lambda_1=5$ and $\lambda_2=-2$\dots and back we go again for the eigenvectors.
        \subsection{Finding eigenvectors}
            \paragraph{For $\lambda_1=5$}
                \[
                    \mathbf{A}-5\mathbf{I}_2
                    =
                    \begin{bmatrix}-3 & 4\\-2 & 6\end{bmatrix}
                    -
                    \begin{bmatrix}5 & 0\\0 & 5\end{bmatrix}
                    =
                    \begin{bmatrix}-3-5 & 4\\-2 & 6-5\end{bmatrix}
                    =
                    \begin{bmatrix}-8 & 4\\-2 & 1\end{bmatrix}
                \]
                Reducing this matrix gives us
                \[
                    \begin{bmatrix}1 & -\frac{1}{2}\\0 & 0\end{bmatrix}
                \]
                So we can take $\mathbf{v}_{1,1}=1$ and $\mathbf{v}_{1,2}=2$ 
                giving us $\mathbf{v}_1=\begin{bmatrix}1\\2\end{bmatrix}$
            \paragraph{For $\lambda_2=-2$}
                Let's imagine, all that work, but with only slight changes.
                Borrowing from my friend pen and paper, we get
                $\mathbf{v}_2=\begin{bmatrix}4\\1\end{bmatrix}$
        \subsection{Questions}
            \paragraph{Part A}
                The stretch factors are simply the eigenvalues we found above.
                So, $\lambda_1=5$ and $\lambda_2=-2$. The stretch directions
                are the eigenvectors we found above.
                To relate this back to how we were presented this before,
                the stretch directions can be thought of as forming the axis
                along the lines $y=2x$ and $y=\frac{1}{4}x$ respectively.
            \paragraph{Part B}
                The eigenvalues are $\lambda_1=5$ and $\lambda_2=-2$.
                The eigenvectors are 
                \[
                    \mathbf{v}_1=\begin{bmatrix}1\\2\end{bmatrix}
                    \,\text{and}\,
                    \mathbf{v}_2=\begin{bmatrix}4\\1\end{bmatrix}
                \]
            \paragraph{Part C}
                Alright, now for the tricky wicket. We want a similarity
                transformation that is similar to a diagonal matrix.
                That was given in the handout as
                \subparagraph{Definition}
                    Two matrices $\mathbf{A},\,\mathbf{B}\,\in\mathbb{R}^{n\times{n}}$ 
                    are similar
                    if $\exists\,\mathbf{C}\in\mathbb{R}^{n\times{n}}$ 
                    where $\exists\,\mathbf{C}^{-1}$ 
                    such that
                    \[
                        \mathbf{B}=\mathbf{C}^{-1}\mathbf{A}\mathbf{C}
                    \]
                \subparagraph{Work}
                    We can take $\mathbf{C}=[\mathbf{v}_1\,\mathbf{v}_2]$ as the matrix,
                    then find our inverse
                    \[
                        \mathbf{C}^{-1}
                        =
                        \begin{bmatrix}1 & 4\\2 & 1\end{bmatrix}^{-1}
                        =
                        \frac{1}{-7}
                        \begin{bmatrix}1 & -4\\-2 & 1\end{bmatrix}
                        =
                        \begin{bmatrix}-\frac{1}{7} & \frac{4}{7}\\\frac{2}{7} & -\frac{1}{7}\end{bmatrix}
                    \]
                    then, plug and chug in to $\mathbf{B}=\mathbf{C}^{-1}\mathbf{A}\mathbf{C}$
                    \[
                        \mathbf{B}
                        =
                        \begin{bmatrix}-\frac{1}{7} & \frac{4}{7}\\\frac{2}{7} & -\frac{1}{7}\end{bmatrix}
                        \begin{bmatrix}-3 & 4\\-2 & 6\end{bmatrix}
                        \begin{bmatrix}1 & 4\\2 & 1\end{bmatrix}
                        =
                        \begin{bmatrix}5 & 0\\0 & -2\end{bmatrix}
                    \]
                    Which is indeed a diagonal matrix.
    \section{Question 4}
        Ok. I went down a few rabbit holes before I got a grasp here.
        The question is to explain why, that if the rows add up to a scalar k,
        then k is an eigenvalue. This made so much more sense once I put together that
        k is my stretch factor, and the vector of all ones is my eigenvector.
        Problems I would have avoided by actually reading the entire question.
        \paragraph{$\mathbf{A}\mathbf{v}_1$}
            \[
                \mathbf{A}\mathbf{v}_1
                =
                \begin{bmatrix}
                    a & b\\
                    c & d
                \end{bmatrix}
                \begin{bmatrix}
                    1\\1
                \end{bmatrix}
                =
                \begin{bmatrix}
                    a+b\\
                    c+d
                \end{bmatrix}
                =
                \begin{bmatrix}
                    k\\
                    k
                \end{bmatrix}
            \]
        \paragraph{$\lambda \mathbf{v}_1$}
            \[
                \lambda \mathbf{v}_1
                =
                k
                \begin{bmatrix}
                    1\\1
                \end{bmatrix}
                =
                \begin{bmatrix}
                    k\\
                    k
                \end{bmatrix}
            \]
        \paragraph{Conclusion}
            As we can see above, $\mathbf{A}\mathbf{v}_1=\lambda \mathbf{v}_1$
            where $\lambda=k$, so by definition k is an eigenvalue of $\mathbf{A}$.
            This is the case as $\mathbf{A}\mathbf{v}_1$ 
            should result in a stretch of $\mathbf{v}_1$ by the factor of k.
\end{document}