\documentclass{article}
\usepackage{fancyhdr} % for pretty formatting
\usepackage{amsmath} % for matrices
\usepackage{amssymb} % for bold text
\usepackage{pgfplots} % for graphs
\usepackage{hyperref} % for hyperlinks
\pgfplotsset{compat=1.18}
\usepackage{enumitem} % for custom lists

\usepackage{tikz}
\usepackage[svgnames]{xcolor}
\usetikzlibrary {positioning}
\definecolor {processblue}{cmyk}{0.96,0,0,0}

\usepackage[math]{cellspace}
\setlength\cellspacetoplimit{3pt}
\setlength\cellspacebottomlimit{3pt} % needed for fractional matrices

\usepackage{lipsum} % For dummy text
\usepackage{cite} % For citations

\pagestyle{fancy}  
\fancyhf{} % Clear all header and footer fields

\usepackage{tcolorbox} % Required for tcolorbox

\newtcolorbox{solutioncheck}{
    colback=green!10, % Background color
    colframe=gray!50, % Frame color
    boxsep=5pt, % Padding
    arc=4pt, % Rounded corners
    title=Checking solution, % Optional title for the aside
    fonttitle=\bfseries, % Title font style
} % for Asides

\lhead{Joshua Dunne}
\rhead{\thepage} % Displays the current page number 
\lfoot{MATH620}
\rfoot{Unit 4}
\cfoot{Homework 5}

\begin{document}
    \section{Question 1}
        \paragraph{Given}
            We're given a set of points that have different offsets
            based on two different basis vectors. First, let's grab
            these points and where they map to in the new basis.
        \subsection{Part A}
            \begin{center}
                \Large
            \begin{tabular}{c c c c}
                $\mathbf{B}_x$ & $\mathbf{B}_y$ & $\mathbf{B}'_x$ & $\mathbf{B}'_y$ \\
                \hline
                1 & 4 & 0 & 1\\
                4 & 7 & 1 & 2\\
                -8 & -5 & -3 & -2\\
                7 & 1 & 1 & 0\\
                $5\frac{1}{2}$ & -5 & 3 & $-\frac{1}{2}$\\
                -6 & $-1\frac{1}{2}$ & $-2\frac{1}{2}$ & -1
            \end{tabular}
            \end{center}
            Where I've denoted $\mathbf{B}$ as the black basis, and $\mathbf{B}'$ as the red.
            These taken together give us plenty of information to find a change of basis.
        \subsection{Part B}
            The points above are of the form $[\mathbf{v}_n]_{\mathbf{B}}\, \rightarrow \,[\mathbf{v}_n]_{\mathbf{B}'}\, \forall{n}$.
            So we can form a system mapping two columns of the first to that of the second.
            \paragraph{The system}
                Let $\mathbf{P}$ be the change of basis we're attempting to find.
                \[
                    \mathbf{P}\begin{bmatrix}1 & 4\\4 & 7\end{bmatrix}
                    =
                    \begin{bmatrix}0 & 1\\1 & 2\end{bmatrix}
                \]
                We need to clear out the left hand side, so take the matrix besides $\mathbf{P}$
                as $\mathbf{A}$, and find the inverse. As $\mathbf{A}\mathbf{A}^{-1}=\mathbf{I}_2$
                \[
                    \mathbf{P}=
                    \begin{bmatrix}0 & 1\\1 & 2\end{bmatrix}
                    \begin{bmatrix}1 & 4\\4 & 7\end{bmatrix}^{-1}
                    =
                    \begin{bmatrix}0 & 1\\1 & 2\end{bmatrix}
                    \frac{1}{9}
                    \begin{bmatrix}7 & -4\\-4 & 1\end{bmatrix}
                    =
                    \frac{1}{9}
                    \begin{bmatrix}4 & -1\\1 & 2\end{bmatrix}
                \]
                \begin{solutioncheck}
                    Let's check this solution by multiplying $\mathbf{P}$ by another point
                    in the original basis, and see if it maps to the correct point in the new basis.
                    \[
                        \mathbf{P}\begin{bmatrix}-8\\-5\end{bmatrix}
                        =
                        \frac{1}{9}
                        \begin{bmatrix}4 & -1\\1 & 2\end{bmatrix}
                        \begin{bmatrix}-8\\-5\end{bmatrix}
                        =
                        \frac{1}{9}
                        \begin{bmatrix}-32 + 5\\-8 - 10\end{bmatrix}
                        =
                        \frac{1}{9}
                        \begin{bmatrix}-27\\-18\end{bmatrix}
                        =
                        \begin{bmatrix}-3\\-2\end{bmatrix}
                    \]
                    This matches the expected output, so our change of basis is correct.
                \end{solutioncheck}
        \subsection{Part C}
            Now we're after a way to go in reverse. We could be super diligent
            and do the same process, but we can also just take the inverse of
            $\mathbf{P}$.
            \[
                \mathbf{P}^{-1}
                =
                \begin{bmatrix}
                    \frac{4}{9} & -\frac{1}{9} \\
                    \frac{1}{9} & \frac{2}{9}
                \end{bmatrix}
            \]
            \begin{solutioncheck}
                Let's check this solution by multiplying $\mathbf{P}^{-1}$ by
                another point in the red basis, and see if it maps to the correct point in the black basis.
                \[
                    \mathbf{P}^{-1}[\mathbf{v}_2]_{\mathbf{B}'}
                    =
                    \begin{bmatrix}
                        2 & 1\\
                        -1 & 4
                    \end{bmatrix}
                    \begin{bmatrix}
                        1\\2
                    \end{bmatrix}
                    =
                    \begin{bmatrix}
                        4\\7
                    \end{bmatrix}
                    =
                    [\mathbf{v}_2]_{\mathbf{B}}
                \]
                Again, seems the intuition is correct. For this point atleast,
                it seems our change of basis is mapping back from black to red.
            \end{solutioncheck}
        \subsection{Part D}
            Ok. I'm reading stretch so, eigenvalues, let's give things a name before we get going
            \paragraph{Given}
                \subparagraph{First line}
                    We are given a stretch factor of $2$ corresponding to the line $y=-\frac{1}{2}x$
                    So that gives us
                    \[
                        T(\begin{bmatrix}2\\-1\end{bmatrix})=\begin{bmatrix}4\\-2\end{bmatrix}
                    \]
                    Where we take $\lambda_1=2$ as the eigenvalue
                \subparagraph{Second line}
                    For the second we're given a stretch factor of $-1$ and a line $y=4x$. From this
                    \[
                        T(\begin{bmatrix}1\\4\end{bmatrix})=\begin{bmatrix}-1\\-4\end{bmatrix}
                    \]
                    Where we take $\lambda_2=-1$ as the eigenvalue
            \paragraph{Forming a system}
                We know that we want a $2x2$ matrix, let's call it $\mathbf{F}$, such that
                \[
                    \mathbf{F}=\begin{bmatrix}a & b\\c & d\end{bmatrix}
                \]
                where
                \begin{center}
                \begin{minipage}{0.49\textwidth}
                    \centering
                    \[
                        \mathbf{F}\begin{bmatrix}2\\-1\end{bmatrix}=\begin{bmatrix}4\\-2\end{bmatrix}
                    \]
                \end{minipage}\hfill
                \begin{minipage}{0.49\textwidth}
                    \centering
                    \[ \mathbf{F}\begin{bmatrix}1\\4\end{bmatrix}=\begin{bmatrix}-1\\-4\end{bmatrix} \]
                \end{minipage}
                \end{center}
                solving for $a$, $b$, $c$, and $d$ we get
                \begin{align*}
                    2a-b&=4\\
                    2c-d&=-2\\
                    a+4b&=-1\\
                    c+4d&=-4
                \end{align*}
                which results $a=\frac{5}{3}$, $b=-\frac{2}{3}$, $c=-\frac{4}{3}$, and $d=-\frac{2}{3}$.
                Giving us a final matrix of
                \[
                    \mathbf{F}=\begin{bmatrix}\frac{5}{3} & -\frac{2}{3} \\ \frac{4}{3} & -\frac{2}{3}\end{bmatrix}
                    =
                    \frac{1}{3}\begin{bmatrix}5 & -2 \\ 4 & -2\end{bmatrix}
                \]
            \paragraph{Mapping red to black}
                \subparagraph{Given}
                    \[
                        [\mathbf{x}]_{\mathbf{r}}
                        =
                        \begin{bmatrix}
                            1\\\frac{1}{2}    
                        \end{bmatrix}
                    \]
                \subparagraph{Work}
                    We need to first map this back to the standard basis, then apply our stretch,
                    then map it back to the red basis. We can borrow our $\mathbf{P}$ from before
                    \[
                        \mathbf{P}[\mathbf{x}]_{\mathbf{r}}=
                        \begin{bmatrix}
                            \frac{4}{9} & -\frac{1}{9} \\
                            \frac{1}{9} & \frac{2}{9}
                        \end{bmatrix}
                        \begin{bmatrix}
                            1\\\frac{1}{2}
                        \end{bmatrix}
                        =
                        \begin{bmatrix}
                            \frac{7}{18}\\\frac{2}{9}
                        \end{bmatrix}
                        =
                        [\mathbf{x}]_{\mathbf{e}}
                    \]
                    And applying our $\mathbf{F}$ stretch
                    \[
                        \mathbf{F}[\mathbf{x}]_{\mathbf{e}}=
                        \frac{1}{3}\begin{bmatrix}5 & -2 \\ 4 & -2\end{bmatrix}
                        \begin{bmatrix}
                            \frac{7}{18}\\\frac{2}{9}
                        \end{bmatrix}
                        =
                        \begin{bmatrix}
                            \frac{1}{2}\\ \frac{10}{27}
                        \end{bmatrix}
                    \]
                    The answer above is in the black basis.
            \paragraph{Mapping black to red}
                \subparagraph{Given}
                    \[
                        [\mathbf{x}]_{\mathbf{e}}
                        =
                        \begin{bmatrix}
                            -3\\3
                        \end{bmatrix}
                    \]
                    We want to apply our transformation, then represent our
                    answer in the red basis.
                \subparagraph{Work}
                    Applying our $\mathbf{F}$ stretch
                    \[
                        \mathbf{F}[\mathbf{x}]_{\mathbf{e}}=
                        \frac{1}{3}\begin{bmatrix}5 & -2 \\ 4 & -2\end{bmatrix}
                        \begin{bmatrix}
                            -3\\3
                        \end{bmatrix}
                        =
                        \begin{bmatrix}
                            -7\\-6
                        \end{bmatrix}
                    \]
                    Now, back to the red basis, we borrow $P^{-1}$ from above
                    \[
                        \mathbf{P}^{-1}\mathbf{F}[\mathbf{x}]_{\mathbf{e}}=
                        \begin{bmatrix}
                            \frac{4}{9} & -\frac{1}{9} \\
                            \frac{1}{9} & \frac{2}{9}
                        \end{bmatrix}
                        \begin{bmatrix}
                            -7\\-6
                        \end{bmatrix}
                        =
                        \begin{bmatrix}
                            -\frac{22}{9}\\-\frac{19}{9}
                        \end{bmatrix}
                    \]
    \section{Question 2}
        Our boy Gauss is back, making our lives difficult again. We're keeping the modes of
        transport we had originally as
        \[
            \beta
            =\left\{
                \begin{bmatrix}
                    3\\1
                \end{bmatrix}
                \begin{bmatrix}
                    1\\2
                \end{bmatrix}
            \right\}
        \]
        And, as I would've guessed\dots changing basis.
        \subsection{Part A}
            \paragraph{Given}
                For one old Uncle Cramer, we're given that in the standard basis
                his house is located at 
                $\left[\mathbf{w}_{\mathbf{e}}\right]=\begin{bmatrix}25\\71\end{bmatrix}$
            \paragraph{Changing basis}
                I want to clarify my understanding for a moment. Above is simply
                analogous to saying that
                \[
                    \begin{bmatrix}1 & 0\\0 & 1\end{bmatrix}
                    \begin{bmatrix}25\\71\end{bmatrix}
                    =
                    \begin{bmatrix}25\\71\end{bmatrix}
                \]
                And our desire to change basis forms a system like so
                \[
                    \begin{bmatrix}3 & 1 \\ 1 & 2\end{bmatrix}
                    \begin{bmatrix}c_1\\c_2\end{bmatrix}
                    =
                    \begin{bmatrix}25\\71\end{bmatrix}
                \]
                Let $\mathbf{P}$ be formed from the columns of our transport basis.
                It's inverse is then
                \[
                    \mathbf{P}^{-1}
                    =
                    \begin{bmatrix}3 & 1 \\ 1 & 2\end{bmatrix}^{-1}
                    =
                    \frac{1}{5}
                    \begin{bmatrix}2 & -1 \\ -1 & 3\end{bmatrix}
                \]
                Lastly, applying our initial vector we get
                \[
                    \mathbf{P}^{-1}
                    \left[\mathbf{w}\right]_{\mathbf{e}}
                    =
                    \left[\mathbf{w}\right]_{\beta}
                    =
                    \frac{1}{5}
                    \begin{bmatrix}2 & -1 \\ -1 & 3\end{bmatrix}
                    \begin{bmatrix}25\\71\end{bmatrix}
                    =
                    \frac{1}{5}
                    \begin{bmatrix}-21\\188\end{bmatrix}
                    =
                    \begin{bmatrix}-\frac{21}{5}\\\frac{188}{5}\end{bmatrix}
                \]
                So, we'd travel $-\frac{21}{5}$ times by hoverboard, and $\frac{188}{5}$ times by magic carpet.
                Not exactly satisfying, but, hey.
        \subsection{Part B}
            \paragraph{Given}
                Now in reverse we're given the location of a museum expressed
                in the transport basis as $\left[\mathbf{v}\right]_{\beta}=\begin{bmatrix}8\\3\end{bmatrix}$
            \paragraph{Changing basis}
                At the risk of turning one mistake in to two, this should be simply the inverse, ie. the original basis
                times the vector in the transport basis.
                \[
                    \mathbf{P}\left[\mathbf{v}\right]_{\beta}
                    =
                    \left[\mathbf{v}\right]_{\mathbf{e}}
                    =
                    \begin{bmatrix}
                        3 & 1 \\ 1 & 2
                    \end{bmatrix}
                    \begin{bmatrix}
                        8\\3
                    \end{bmatrix}
                    =
                    \begin{bmatrix}
                        27\\14
                    \end{bmatrix}
                \]

\end{document}