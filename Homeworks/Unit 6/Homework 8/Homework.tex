\documentclass{article}
\usepackage{fancyhdr} % for pretty formatting
\usepackage{amsmath} % for matrices
\usepackage{amssymb} % for bold text
\usepackage{pgfplots} % for graphs
\usepackage{hyperref} % for hyperlinks
\pgfplotsset{compat=1.18}
\usepackage{enumitem} % for custom lists

\usepackage{tikz}
\usepackage[svgnames]{xcolor}
\usetikzlibrary {positioning}
\definecolor {processblue}{cmyk}{0.96,0,0,0}

\usepackage[math]{cellspace}
\setlength\cellspacetoplimit{3pt}
\setlength\cellspacebottomlimit{3pt} % needed for fractional matrices

\usepackage{lipsum} % For dummy text
\usepackage{cite} % For citations

\pagestyle{fancy}  
\fancyhf{} % Clear all header and footer fields

\usepackage{tcolorbox} % Required for tcolorbox

\newtcolorbox{solutioncheck}{
    colback=green!10, % Background color
    colframe=gray!50, % Frame color
    boxsep=5pt, % Padding
    arc=4pt, % Rounded corners
    title=Checking solution, % Optional title for the aside
    fonttitle=\bfseries, % Title font style
} % for Asides

\lhead{Joshua Dunne}
\rhead{\thepage} % Displays the current page number 
\lfoot{MATH620}
\rfoot{Unit 6}
\cfoot{Homework 8}

\begin{document}
    \section{Question 1}
        \paragraph{Statement}
            A linear transformation $\mathbf{T}$ transforms
            \[
                \begin{bmatrix}2\\1\end{bmatrix}
                \,\text{to}\,
                \begin{bmatrix}-1\\-5\end{bmatrix}
                \quad\text{and}\quad
                \begin{bmatrix}-3\\-2\end{bmatrix}
                \,\text{to}\,
                \begin{bmatrix}-15\\-10\end{bmatrix}.
            \]
            We're asked to find where a similar matrix is transformed to,
            without calculating the linear transformation $\mathbf{A}$ explicitly.
        \paragraph{Givens}
            \[
                \text{let}\quad\mathbf{u}=\begin{bmatrix}2\\1\end{bmatrix},\,
                \mathbf{v}=\begin{bmatrix}-3\\-2\end{bmatrix},\,
                \mathbf{w}=\begin{bmatrix}16\\10\end{bmatrix}
            \]
            We know from above that
            \[
                T(\mathbf{u})=\begin{bmatrix}-1\\-5\end{bmatrix}
                \quad\text{and}\quad
                T(\mathbf{v})=\begin{bmatrix}-15\\-10\end{bmatrix}
            \]
            We also know from previous investigation and work done in class that
            the linear transformations exhibit the additive property
            \[
                T(c_1\mathbf{u}+c_2\mathbf{v})=c_1T(\mathbf{u})+c_2T(\mathbf{v})
            \]
        \paragraph{Work}
            So, we can find the coefficients $c_1$ and $c_2$ such that
            \[
                \begin{bmatrix}16\\10\end{bmatrix}
                =c_1\begin{bmatrix}2\\1\end{bmatrix}
                +c_2\begin{bmatrix}-3\\-2\end{bmatrix}
            \]
            Using the usual methods, we get $c_1=2$ and $c_2=-4$.
            We can then relate this back and see that
            \[
                T(\mathbf{w})=T(2\mathbf{u}+4\mathbf{v})
                =2T(\mathbf{u})-4T(\mathbf{v})
                =2\begin{bmatrix}-1\\-5\end{bmatrix}
                -4\begin{bmatrix}-15\\-10\end{bmatrix}
            \]
        \paragraph{Solution}
            Calculating the above, we get
            \[
                2\begin{bmatrix}-1\\-5\end{bmatrix}
                -
                4\begin{bmatrix}-15\\-10\end{bmatrix}
                =
                \begin{bmatrix}58\\30\end{bmatrix}
            \]
            So. Without finding the actual linear transformation,
            we can use the properties of linear transformations to avoid 
            actually calculating the transformation matrix. Though
            to have had enough information to do this, we would have had to
            known the images of two linearly independent vectors.
    \section{Question 2}
        \subsection{Statement}
            Ok. This time we want the eigenvalues and eigenvectors
            of the matrix. We then want to interpret what this means geometrically.
        \subsection{Givens}
            \[
                \mathbf{A}=\begin{bmatrix}1 & 0 & 1\\0 & 4 & 0\\12 & 2 & 2\end{bmatrix}
            \]
        \subsection{Work}
            We were given on of the eigenvalues, $\lambda_1=4$. Using this we
            could have used division to find the others, but\dots
            We can also just calculate the characteristic polynomial
            using a calculator and factor back.
            That's alot of terms to avoid making a mistake with.
            \[
                det(\mathbf{A}-\lambda\mathbf{I}_3)
                =
                (5-x)(x-4)(x+2)
            \]
            \paragraph{For $\lambda_1=4$}
                \[
                    \mathbf{A}-\lambda_1\mathbf{I}_3
                    =
                    \begin{bmatrix}-3 & 0 & 1\\0 & 0 & 0\\12 & 2 & -2\end{bmatrix}
                    \rightarrow
                    \begin{bmatrix}
                        1 & 0 & -\frac{1}{3}\\0 & 1 & 1\\0 & 0 & 0
                    \end{bmatrix}
                \]
                From this we take
                \begin{align*}
                    x_1-\frac{1}{3}x_3&=0\\
                    x_2+x_3&=0
                \end{align*}
                We let $x_3=s$ as our free variable, let $s=3$, and find
                \[
                    \mathbf{v}_1
                    =
                    \begin{bmatrix}1\\-3\\3\end{bmatrix}
                \]
            \paragraph{For $\lambda_2=5$}
                \[
                    \mathbf{A}-\lambda_1\mathbf{I}_3
                    =
                    \begin{bmatrix}-4 & 0 & 1\\0 & -1 & 0\\12 & 2 & -3\end{bmatrix}
                    \rightarrow
                    \begin{bmatrix}
                        1 & 0 & -\frac{1}{4}\\0 & 1 & 0\\0 & 0 & 0
                    \end{bmatrix}
                \]
                From this we take
                \begin{align*}
                    x_1-\frac{1}{4}x_3&=0\\
                    x_2&=0
                \end{align*}
                We let $x_3=s$ as our free variable, let $s=4$, and find
                \[
                    \mathbf{v}_2
                    =
                    \begin{bmatrix}1\\0\\4\end{bmatrix}
                \]
            \paragraph{For $\lambda_3=-2$}
                \[
                    \mathbf{A}-\lambda_1\mathbf{I}_3
                    =
                    \begin{bmatrix}3 & 0 & 1\\0 & 6 & 0\\12 & 2 & 4\end{bmatrix}
                    \rightarrow
                    \begin{bmatrix}
                        1 & 0 & \frac{1}{3}\\0 & 1 & 0\\0 & 0 & 0
                    \end{bmatrix}
                \]
                From this we take
                \begin{align*}
                    x_1+\frac{1}{3}x_3&=0\\
                    x_2&=0
                \end{align*}
                We let $x_3=s$ as our free variable, let $s=3$, and find
                \[
                    \mathbf{v}_3
                    =
                    \begin{bmatrix}1\\0\\-3\end{bmatrix}
                \]
        \subsection{Solution}
            So, we have the eigenvalues and eigenvectors as follows:
            \begin{align*}
                \lambda_1=4, \mathbf{v}_1=\begin{bmatrix}1\\-3\\3\end{bmatrix}; \quad
                \lambda_2=5, \mathbf{v}_2=\begin{bmatrix}1\\0\\4\end{bmatrix}; \quad
                \lambda_3=-2, \mathbf{v}_3=\begin{bmatrix}1\\0\\-3\end{bmatrix}
            \end{align*}
            \paragraph{Interpretation}
                So, geometrically, \textbf{if} a point lies on one of the lines
                defined by the eigenvectors, then when the transformation is applied,
                the point will be scaled by the corresponding eigenvalue. 
                Also, \textbf{if} a point does not lie along a line, it will be stretched
                by some combination of the eigenvalues in the directions of the eigenvectors.
                \subparagraph{Quick example}
                    We know that the point
                    \[
                        3\begin{bmatrix}1\\-3\\3\end{bmatrix}
                        =
                        \begin{bmatrix}3\\-9\\9\end{bmatrix}
                    \]
                    lies along the line defined by $\mathbf{v}_1$.
                    Applying the transformation, we get
                    \[
                        \mathbf{A}
                        \begin{bmatrix}3\\-9\\9\end{bmatrix}
                        =
                        3\mathbf{A}
                        \begin{bmatrix}1\\-3\\3\end{bmatrix}
                        =
                        3\cdot4
                        \begin{bmatrix}1\\-3\\3\end{bmatrix}
                        =
                        \begin{bmatrix}12\\-36\\36\end{bmatrix}
                        =
                        \lambda_1
                        \begin{bmatrix}3\\-9\\9\end{bmatrix}
                    \]
                    So, taking a point along that line, we see that it is scaled
                    by $\lambda_1=4$ as expected.
\end{document}