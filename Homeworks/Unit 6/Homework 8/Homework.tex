\documentclass{article}
\usepackage{fancyhdr} % for pretty formatting
\usepackage{amsmath} % for matrices
\usepackage{amssymb} % for bold text
\usepackage{pgfplots} % for graphs
\usepackage{hyperref} % for hyperlinks
\pgfplotsset{compat=1.18}
\usepackage{enumitem} % for custom lists

\usepackage{tikz}
\usepackage[svgnames]{xcolor}
\usetikzlibrary {positioning}
\definecolor {processblue}{cmyk}{0.96,0,0,0}

\usepackage[math]{cellspace}
\setlength\cellspacetoplimit{3pt}
\setlength\cellspacebottomlimit{3pt} % needed for fractional matrices

\usepackage{lipsum} % For dummy text
\usepackage{cite} % For citations

\pagestyle{fancy}  
\fancyhf{} % Clear all header and footer fields

\usepackage{tcolorbox} % Required for tcolorbox

\newtcolorbox{solutioncheck}{
    colback=green!10, % Background color
    colframe=gray!50, % Frame color
    boxsep=5pt, % Padding
    arc=4pt, % Rounded corners
    title=Checking solution, % Optional title for the aside
    fonttitle=\bfseries, % Title font style
} % for Asides

\lhead{Joshua Dunne}
\rhead{\thepage} % Displays the current page number 
\lfoot{MATH620}
\rfoot{Unit 6}
\cfoot{Homework 8}

\begin{document}
    \section{Question 1}
        \paragraph{Statement}
            A linear transformation $\mathbf{T}$ transforms
            \[
                \begin{bmatrix}2\\1\end{bmatrix}
                \,\text{to}\,
                \begin{bmatrix}-1\\-5\end{bmatrix}
                \quad\text{and}\quad
                \begin{bmatrix}-3\\-2\end{bmatrix}
                \,\text{to}\,
                \begin{bmatrix}-15\\-10\end{bmatrix}.
            \]
            We're asked to find where a similar matrix is transformed to,
            without calculating the linear transformation $\mathbf{A}$ explicitly.
        \paragraph{Givens}
            \[
                \text{let}\quad\mathbf{u}=\begin{bmatrix}2\\1\end{bmatrix},\,
                \mathbf{v}=\begin{bmatrix}-3\\-2\end{bmatrix},\,
                \mathbf{w}=\begin{bmatrix}16\\10\end{bmatrix}
            \]
            We know from above that
            \[
                T(\mathbf{u})=\begin{bmatrix}-1\\-5\end{bmatrix}
                \quad\text{and}\quad
                T(\mathbf{v})=\begin{bmatrix}-15\\-10\end{bmatrix}
            \]
            We also know from previous investigation and work done in class that
            the linear transformations exhibit the additive property
            \[
                T(c_1\mathbf{u}+c_2\mathbf{v})=c_1T(\mathbf{u})+c_2T(\mathbf{v})
            \]
        \paragraph{Work}
            So, we can find the coefficients $c_1$ and $c_2$ such that
            \[
                \begin{bmatrix}16\\10\end{bmatrix}
                =c_1\begin{bmatrix}2\\1\end{bmatrix}
                +c_2\begin{bmatrix}-3\\-2\end{bmatrix}
            \]
            Using the usual methods, we get $c_1=2$ and $c_2=-4$.
            We can then relate this back and see that
            \[
                T(\mathbf{w})=T(2\mathbf{u}+4\mathbf{v})
                =2T(\mathbf{u})-4T(\mathbf{v})
                =2\begin{bmatrix}-1\\-5\end{bmatrix}
                -4\begin{bmatrix}-15\\-10\end{bmatrix}
            \]
        \paragraph{Solution}
            Calculating the above, we get
            \[
                2\begin{bmatrix}-1\\-5\end{bmatrix}
                -
                4\begin{bmatrix}-15\\-10\end{bmatrix}
                =
                \begin{bmatrix}58\\30\end{bmatrix}
            \]
            So. Without finding the actual linear transformation,
            we can use the properties of linear transformations to avoid 
            actually calculating the transformation matrix. Though
            to have had enough information to do this, we would have had to
            known the images of two linearly independent vectors.
    \section{Question 2}
        \subsection{Statement}
            Ok. This time we want the eigenvalues and eigenvectors
            of the matrix. We then want to interpret what this means geometrically.
        \subsection{Givens}
            \[
                \mathbf{A}=\begin{bmatrix}1 & 0 & 1\\0 & 4 & 0\\12 & 2 & 2\end{bmatrix}
            \]
        \subsection{Work}
            We were given on of the eigenvalues, $\lambda_1=4$. Using this we
            could have used division to find the others, but\dots
            We can also just calculate the characteristic polynomial
            using a calculator and factor back.
            That's alot of terms to avoid making a mistake with.
            \[
                det(\mathbf{A}-\lambda\mathbf{I}_3)
                =
                (5-x)(x-4)(x+2)
            \]
            \paragraph{For $\lambda_1=4$}
                \[
                    \mathbf{A}-\lambda_1\mathbf{I}_3
                    =
                    \begin{bmatrix}-3 & 0 & 1\\0 & 0 & 0\\12 & 2 & -2\end{bmatrix}
                    \rightarrow
                    \begin{bmatrix}
                        1 & 0 & -\frac{1}{3}\\0 & 1 & 1\\0 & 0 & 0
                    \end{bmatrix}
                \]
                From this we take
                \begin{align*}
                    x_1-\frac{1}{3}x_3&=0\\
                    x_2+x_3&=0
                \end{align*}
                We let $x_3=s$ as our free variable, let $s=3$, and find
                \[
                    \mathbf{v}_1
                    =
                    \begin{bmatrix}1\\-3\\3\end{bmatrix}
                \]
            \paragraph{For $\lambda_2=5$}
                \[
                    \mathbf{A}-\lambda_1\mathbf{I}_3
                    =
                    \begin{bmatrix}-4 & 0 & 1\\0 & -1 & 0\\12 & 2 & -3\end{bmatrix}
                    \rightarrow
                    \begin{bmatrix}
                        1 & 0 & -\frac{1}{4}\\0 & 1 & 0\\0 & 0 & 0
                    \end{bmatrix}
                \]
                From this we take
                \begin{align*}
                    x_1-\frac{1}{4}x_3&=0\\
                    x_2&=0
                \end{align*}
                We let $x_3=s$ as our free variable, let $s=4$, and find
                \[
                    \mathbf{v}_2
                    =
                    \begin{bmatrix}1\\0\\4\end{bmatrix}
                \]
            \paragraph{For $\lambda_3=-2$}
                \[
                    \mathbf{A}-\lambda_1\mathbf{I}_3
                    =
                    \begin{bmatrix}3 & 0 & 1\\0 & 6 & 0\\12 & 2 & 4\end{bmatrix}
                    \rightarrow
                    \begin{bmatrix}
                        1 & 0 & \frac{1}{3}\\0 & 1 & 0\\0 & 0 & 0
                    \end{bmatrix}
                \]
                From this we take
                \begin{align*}
                    x_1+\frac{1}{3}x_3&=0\\
                    x_2&=0
                \end{align*}
                We let $x_3=s$ as our free variable, let $s=3$, and find
                \[
                    \mathbf{v}_3
                    =
                    \begin{bmatrix}1\\0\\-3\end{bmatrix}
                \]
        \subsection{Solution}
            So, we have the eigenvalues and eigenvectors as follows:
            \begin{align*}
                \lambda_1=4, \mathbf{v}_1=\begin{bmatrix}1\\-3\\3\end{bmatrix}; \quad
                \lambda_2=5, \mathbf{v}_2=\begin{bmatrix}1\\0\\4\end{bmatrix}; \quad
                \lambda_3=-2, \mathbf{v}_3=\begin{bmatrix}1\\0\\-3\end{bmatrix}
            \end{align*}
            \paragraph{Interpretation}
                So, geometrically, \textbf{if} a point lies on one of the lines
                defined by the eigenvectors, then when the transformation is applied,
                the point will be scaled by the corresponding eigenvalue. 
                Also, \textbf{if} a point does not lie along a plane, it will be stretched
                by some combination of the eigenvalues in the directions of the eigenvectors.
                \subparagraph{Quick example}
                    We know that the point
                    \[
                        3\begin{bmatrix}1\\-3\\3\end{bmatrix}
                        =
                        \begin{bmatrix}3\\-9\\9\end{bmatrix}
                    \]
                    lies along the plane defined by $\mathbf{v}_1$.
                    Applying the transformation, we get
                    \[
                        \mathbf{A}
                        \begin{bmatrix}3\\-9\\9\end{bmatrix}
                        =
                        3\mathbf{A}
                        \begin{bmatrix}1\\-3\\3\end{bmatrix}
                        =
                        3\cdot4
                        \begin{bmatrix}1\\-3\\3\end{bmatrix}
                        =
                        \begin{bmatrix}12\\-36\\36\end{bmatrix}
                        =
                        \lambda_1
                        \begin{bmatrix}3\\-9\\9\end{bmatrix}
                    \]
                    So, taking a point along that plane, we see that it is scaled
                    by $\lambda_1=4$ as expected. We could explore this a little further,
                    but it's going to come up again later, so, I'll save that for then.
    \section{Question 3}
        \subsection{Statement}
            Ok, we're back at it, 
            but this time we're given a tranformation 
            $T\,:\,\mathbb{R}^3\rightarrow\mathbb{R}^3$
            that stretches
            \[
                \frac{1}{4}\:\text{in}\:\begin{bmatrix}-1\\0\\2\end{bmatrix}
                ,\:
                -3\:\text{in}\:\begin{bmatrix}0\\-1\\5\end{bmatrix}
                ,\:
                -3\:\text{in}\:\begin{bmatrix}1\\-1\\9\end{bmatrix}
            \]
            We can take these as eigenvalues and eigenvectors
            \begin{align*}
                \lambda_1=\frac{1}{4}, \mathbf{v}_1=\begin{bmatrix}-1\\0\\2\end{bmatrix}; \quad
                \lambda_2=-3, \mathbf{v}_2=\begin{bmatrix}0\\-1\\5\end{bmatrix}; \quad
                \lambda_3=-3, \mathbf{v}_3=\begin{bmatrix}1\\-1\\9\end{bmatrix}
            \end{align*}
        \subsection{Question by parts}
            \paragraph{Part A}
                We want to find why the vector
                \[
                    \text{let}\:\mathbf{A}=\begin{bmatrix}3\\-5\\37\end{bmatrix}
                \]
                stretches by a factor of $-3$.
                \subparagraph{Work}
                    We can express $\mathbf{A}$ as a linear combination
                    of the eigenvectors.
                    \[
                        \begin{bmatrix}3\\-5\\37\end{bmatrix}
                        =
                        c_1\begin{bmatrix}-1\\0\\2\end{bmatrix}
                        +
                        c_2\begin{bmatrix}0\\-1\\5\end{bmatrix}
                        +
                        c_3\begin{bmatrix}1\\-1\\9\end{bmatrix}
                    \]
                    Solving this system, we find that
                    \[
                        c_1=0, c_2=2, c_3=3
                    \]
                    So, we can see that
                    \[
                        \mathbf{A}
                        =
                        0\begin{bmatrix}-1\\0\\2\end{bmatrix}
                        +
                        2\begin{bmatrix}0\\-1\\5\end{bmatrix}
                        +
                        3\begin{bmatrix}1\\-1\\9\end{bmatrix}
                    \]
                \subparagraph{Solution}
                    Because $\mathbf{A}$ is a linear combination of two
                    eigenvectors that have the same eigenvalue $\lambda_2=\lambda_3=-3$,
                    when the transformation is applied, both components will
                    be scaled by $-3$, resulting in the entire vector being scaled
                    by $-3$.
            \paragraph{Part B}
                Again, without finding the actual transformation, we're going to do this
                three more times. We follow the same methodology as above,
                solving the same system of equations, but with the given vector
                as the augment.
                \subparagraph{Subpart i}
                    \[
                        \text{given}\:\mathbf{A}=\begin{bmatrix}3\\0\\6\end{bmatrix}
                        \rightarrow
                        \begin{bmatrix}1&0&0&-3\\0&1&0&0\\0&0&1&0\end{bmatrix}
                    \]
                    So, we can express the given vector as lying on the plane of $\mathbf{v}_1$.
                    Given such, we know that it stretches by $\frac{1}{4}$.
                \subparagraph{Subpart ii}
                    \[
                        \text{given}\:\mathbf{A}=
                        \begin{bmatrix}\frac{1}{2}\\-\frac{1}{2}\\\frac{9}{2}\end{bmatrix}
                        \rightarrow
                        \begin{bmatrix}1&0&0&0\\0&1&0&0\\0&0&1&\frac{1}{2}\end{bmatrix}
                    \]
                    So, we can express the given vector as lying on the plane of $\mathbf{v}_3$.
                    Given such, we know that it stretches by $-3$.
                \subparagraph{Subpart iii}
                    \[
                        \text{given}\:\mathbf{A}=
                        \begin{bmatrix}-1\\0\\0\end{bmatrix}
                        \rightarrow
                        \begin{bmatrix}
                            1&0&0&\frac{2}{3}\\
                            0&1&0&\frac{1}{3}\\
                            0&0&1&-\frac{1}{3}
                        \end{bmatrix}
                    \]
                    This is the tricky one. Because we need to combine the spans of the
                    eigenvectors to reach the point, it scales as a combination of the three.
                    We can represent this like so
                    \[
                        \mathbf{A}
                        =
                        \frac{2}{3}\mathbf{v}_1
                        +\frac{1}{3}\mathbf{v}_2
                        -\frac{1}{3}\mathbf{v}_3
                    \]
                    Our current thinking is that it scales like this
                    \[
                        \lambda_1\frac{2}{3}\mathbf{v}_1
                        -
                        \lambda_2\frac{1}{3}\mathbf{v}_2
                        +
                        \lambda_3\frac{1}{3}\mathbf{v}_3
                    \]
                    But as to a perfectly clear picture as to what's going on,
                    I'll leave this until class and hopefully clarify.
\end{document}