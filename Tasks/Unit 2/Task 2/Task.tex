\documentclass{article}
\usepackage{fancyhdr}
\usepackage{amsmath}
\usepackage{amssymb} % For the R symbol

\usepackage{lipsum} % For dummy text

\pagestyle{fancy}
\fancyhf{} % Clear all header and footer fields

\lhead{Joshua Dunne}
\rhead{\thepage} % Displays the current page number
\lfoot{MATH620}
\rfoot{Unit 2}
\cfoot{Task 2}

\begin{document}
\section{Systems of Equations}
    The goal is to create systems of linear equations that have no solution, one unique solution, and infinitely many solutions.
    \begin{itemize}
        \item \textbf{No solution (Inconsistent System)}
        
        We can create an inconsistent system by having two parallel lines. For example:
        \begin{align*}
            x + 2y &= 3 \\
            x + 2y &= 5
        \end{align*}
        If we subtract the first equation from the second, we get $0 = 2$, which is a contradiction. This means there is no pair of $(x, y)$ that can satisfy both equations simultaneously.
        
        \item \textbf{One unique solution (Consistent and Independent)}
        
        A system with one solution consists of two lines that intersect at a single point. The vectors representing the equations must be linearly independent.
        \begin{align*}
            2x + 3y &= 7 \\
            x - y &= 1
        \end{align*}
        We can solve this using substitution or elimination. From the second equation, $x = y + 1$. Substituting into the first:
        \begin{align*}
            2(y+1) + 3y &= 7 \\
            2y + 2 + 3y &= 7 \\
            5y &= 5 \\
            y &= 1
        \end{align*}
        Then $x = 1 + 1 = 2$. The unique solution is $(2, 1)$.
        
        \item \textbf{Infinitely many solutions (Consistent and Dependent)}
        
        This occurs when the equations in the system are multiples of each other, essentially representing the same line.
        \begin{align*}
            x + 2y &= 4 \\
            2x + 4y &= 8
        \end{align*}
        The second equation is just the first equation multiplied by 2. Any point $(x, y)$ that lies on the line $x + 2y = 4$ is a solution. We can express the solutions in terms of a parameter, say $t$. Let $y=t$, where $t \in \mathbb{R}$. Then $x = 4 - 2t$. The solution set is all points of the form $(4 - 2t, t)$.
    \end{itemize}

\section{To check whether using rref}
\begin{itemize}
    \item No solution
    \[
    \begin{bmatrix}
        \begin{array}{cc|c}
            1 & 2 &  3 \\
            1 & 2 &  5 \\
        \end{array}
    \end{bmatrix}
    \xrightarrow{}
    \begin{bmatrix}
        \begin{array}{cc|c}
            1 & 2 &  3 \\
            0 & 0 &  2 \\
        \end{array}
    \end{bmatrix}
    \]
    The second row translates to $0 = 2$, which is a contradiction, confirming no solution.
    If we ever run in to a contradiction like this, we know there is no solution.

    \item One solution
    \[
    \begin{bmatrix}
        \begin{array}{cc|c}
            2 & 3 &  7 \\
            1 & 1 &  1 \\
        \end{array}
    \end{bmatrix}
    \xrightarrow{}
    \begin{bmatrix}     
        \begin{array}{cc|c}
            1 & 0 &  -1 \\
            0 & 1 &  3 \\
        \end{array} 
    \end{bmatrix}
    \]
    Here we have no codependencies. That is, we do not
    have to presume a value for one variable to find the other.
    This means there is one unique solution, which is $(-1, 3)$.
    \item Infinitely many solutions
    \[
    \begin{bmatrix}
        \begin{array}{cc|c}
            1 & 2 &  4 \\
            2 & 4 &  8 \\
        \end{array}
    \end{bmatrix}
    \xrightarrow{}
    \begin{bmatrix}
        \begin{array}{cc|c}
            1 & 2 &  4 \\
            0 & 0 &  0 \\
        \end{array}
    \end{bmatrix}
    \]
    So. One equation. Two variables.
    We can let $y=t$, where $t \in \mathbb{R}$
    Then $x = 4 - 2t$. The solution set is all points of the form $(4 - 2t, t)$.
\end{itemize}

\end{document}