\documentclass{article}
\usepackage{fancyhdr} % for pretty formatting
\usepackage{amsmath} % for matrices
\usepackage{amssymb} % for bold text
\usepackage{lipsum} % For dummy text

\pagestyle{fancy}
\fancyhf{} % Clear all header and footer fields

\lhead{Joshua Dunne}
\rhead{\thepage} % Displays the current page number
\lfoot{MATH620}
\cfoot{Task 2}

\begin{document}
\section{Intuition}
Time for a little bit of further digging. We ascertained that we could reach
old man Guass's house at the given offset, but, is there anywhere where we would
be unable to reach him? Put another way. What is the span of the two column vectors
representing the magic carpet and hoverboard?
\section{Details}
Examine
\begin{center}
$
span(\begin{bmatrix}3 \\ 1 \end{bmatrix},\begin{bmatrix}1 \\ 2 \end{bmatrix})
={c_1}\begin{bmatrix}3 \\ 1 \end{bmatrix} + {c_2}\begin{bmatrix}1 \\ 2 \end{bmatrix}
$
where $c_1$, $c_2 \in \mathbb{R}$
\end{center}

We can see that both are linearlly indepent as
\[det(\begin{bmatrix}3 & 1 \\ 1 & 2\end{bmatrix}) = 5 \ne 0\]
And from this devise that our two methods of transport do indeed span $\mathbb{R}^2$
\section{Conclusion}
As the two columns vectors representing the hoverboard and magic carpet do indeed
span all of $\mathbb{R}^2$, there is nowhere on the plane that old man Guass can hide.
We'll always be able to find him with the help of our hoverboard and magic carpet.
\end{document}