\documentclass{article}
\usepackage{fancyhdr} % for pretty formatting
\usepackage{amsmath} % for matrices
\usepackage{amssymb} % for bold text
\usepackage{pgfplots} % for graphs
\usepackage{hyperref} % for hyperlinks
\pgfplotsset{compat=1.18}
\usepackage{enumitem} % for custom lists

\usepackage{tikz}
\usetikzlibrary {positioning}
\definecolor {processblue}{cmyk}{0.96,0,0,0}

\usepackage[svgnames]{xcolor}

\usepackage{lipsum} % For dummy text
\usepackage{cite} % For citations

\pagestyle{fancy}  
\fancyhf{} % Clear all header and footer fields

\usepackage{tcolorbox} % Required for tcolorbox

\newtcolorbox{solutioncheck}{
    colback=green!10, % Background color
    colframe=gray!50, % Frame color
    boxsep=5pt, % Padding
    arc=4pt, % Rounded corners
    title=Checking solution, % Optional title for the aside
    fonttitle=\bfseries, % Title font style
} % for Asides

\lhead{Joshua Dunne}
\rhead{\thepage} % Displays the current page number 
\lfoot{MATH620}
\rfoot{Unit 4}
\cfoot{Homework 5}

\begin{document}
    \section{Part 2}
        We're composing two sets of vectors for a person traveling
        between $A\rightarrow C$ and $C\rightarrow C$.
        We've already done some work here getting a feel for what
        the vectors actually mean, so, let's try to describe the sets
        $S_{AC}$ and $S_{CC}$.
        \subsection[Composing A to C]{For $S_{AC}$}
            So. We can break this down. Really. Once we're at
            $C$ things are the same, so, we can borrow from the
            work we actually did first\dots
            Avoiding drawing a\dots fine, I'll do it.
            \footnote{Borrowing from https://tex.stackexchange.com/questions/57152/how-to-draw-graphs-in-latex}
            \begin {center}
            \begin {tikzpicture}[-latex ,auto ,node distance =4 cm and 5cm ,on grid ,
                semithick ,
                state/.style ={ circle ,top color =white , bottom color = processblue!20 ,
                draw,processblue , text=blue , minimum width =1 cm}]
                \node[state] (A){$A$};
                \node[state] (B) [above=of A]{$B$};
                \node[state] (C) [right=of B]{$C$};
                \node[state] (D) [below=of C]{$D$};
                \path (A) edge node[left] {$C_1$} (B);
                \path (B) edge node[above] {$C_2$} (C);
                \path (C) edge node[left] {$C_3$} (D);
                \path (D) edge node[above] {$C_5$} (A);
                \path (C) edge node[above left] {$C_4$} (A);

            \end{tikzpicture}
            \end{center}
            We take as the number of moves past each camera $C_n$ the
            corresponding entry in the column vector
            \[
            \begin{bmatrix}c_1\\c_2\\c_3\\c_4\\c_5\end{bmatrix}
            \]
            \paragraph{Part A}
                Going to do the work for B here and restate the answer later,
                as having shown it here gives us an easy way by which to generate
                $4$ members of the set.
                \[
                S_{AC}=
                \begin{bmatrix}1\\1\\0\\0\\0\end{bmatrix}
                +
                c_1\begin{bmatrix}1\\1\\0\\1\\0\end{bmatrix}
                +
                c_2\begin{bmatrix}1\\1\\1\\0\\1\end{bmatrix}
                \]
                It doesn't seem worth the effort to illustrate this. We're
                composing all the possible set of vectors that
                \begin{enumerate}
                    \item Go from $A\rightarrow B\rightarrow C$ to begin with
                    \item Travel from $C$ to $C$ via $C\rightarrow D\rightarrow A\rightarrow B\rightarrow C$
                    \item Travel from $C$ to $C$ via $C\rightarrow A\rightarrow B\rightarrow C$
                \end{enumerate}
                This composes the set nicely. We can plug in a couple of coefficients for each, giving
                \begin{enumerate}
                    \item Letting $c_1=c_2=0$ gives $\langle 1, 1, 0, 0, 0\rangle \in S_{AC}$
                    \item Letting $c_1=1$ and $c_2=0$ gives $\langle 2, 2, 1, 0, 1\rangle \in S_{AC}$
                    \item Letting $c_1=0$ and $c_2=1$ gives $\langle 2, 2, 0, 1, 0\rangle \in S_{AC}$
                    \item Letting $c_1=1$ and $c_2=1$ gives $\langle 3, 3, 1, 1, 1\rangle \in S_{AC}$
                \end{enumerate}
                We can understand $\langle 3, 3, 1, 1, 1\rangle$ as taking each of the separate paths we've defined
            \paragraph{Part B}
                Restating as above
                \[
                S_{AC}=
                \begin{bmatrix}1\\1\\0\\0\\0\end{bmatrix}
                +
                c_1\begin{bmatrix}1\\1\\0\\1\\0\end{bmatrix}
                +
                c_2\begin{bmatrix}1\\1\\1\\0\\1\end{bmatrix}
                \text{ where }
                c_1, c_2 \in \mathbb{Z}^{*}
                \]
                we simply need to add a restriction here, as,
                negative and partial movements are not defined.
        \subsection[Composing C to C]{For $S_{CC}$}
            We've already done all the work here, and I think the above
            explanations are sufficient, so, I'll be brief.
            \paragraph{Part A}
                \[
                S_{CC}=
                c_1\begin{bmatrix}1\\1\\0\\1\\0\end{bmatrix}
                +
                c_2\begin{bmatrix}1\\1\\1\\0\\1\end{bmatrix}
                \text{ where }
                c_1, c_2 \in \mathbb{Z}^{*}
                \]
                \begin{enumerate}
                    \item Letting $c_1=c_2=0$ gives $\langle 0, 0, 0, 0, 0\rangle \in S_{CC}$, the trivial solution
                    \item Letting $c_1=1$ and $c_2=0$ gives $\langle 1, 1, 0, 1, 0\rangle \in S_{CC}$
                    \item Letting $c_1=0$ and $c_2=1$ gives $\langle 1, 1, 1, 0, 1\rangle \in S_{CC}$
                    \item Letting $c_1=1$ and $c_2=1$ gives $\langle 2, 2, 1, 1, 1\rangle \in S_{CC}$
                \end{enumerate}
            \paragraph{Part B}
                Restating as above
                \[
                S_{CC}=
                c_1\begin{bmatrix}1\\1\\0\\1\\0\end{bmatrix}
                +
                c_2\begin{bmatrix}1\\1\\1\\0\\1\end{bmatrix}
                \text{ where }
                c_1, c_2 \in \mathbb{Z}^{*}
                \]
\end{document}