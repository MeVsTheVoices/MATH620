\documentclass{article}
\usepackage{fancyhdr}
\usepackage{amsmath}

\usepackage{lipsum} % For dummy text

\pagestyle{fancy}
\fancyhf{} % Clear all header and footer fields

\lhead{Joshua Dunne}
\rhead{\thepage} % Displays the current page number
\lfoot{MATH620}
\cfoot{Task 1}

\begin{document}
\section{Intuition}
The task we are given here is to take two modes of transportation, a hoverboard that moves
3 units east and 1 north, and a magic carpet that movies 1 unit east and 2 north.
From this we are to determine whether or not some combination of the these two
modes of transportation will allow us to arrive at old man Guass's house at
107 units east and 64 miles north of our original position.
\section{Further Details}
We can represent this mathematically using column vectors and coefficients.
\[
T_C = \begin{bmatrix}
3 \\ 1 
\end{bmatrix}
,
T_M = \begin{bmatrix}
1 \\ 2
\end{bmatrix}
,
T = \begin{bmatrix}
107 \\ 64
\end{bmatrix}
\]
\[
T = {c_1}{T_C}+{c_2}{T_M}
\]
From here we simply want to find two corresponding values for the 
coefficients ${c_1}$ and ${c_2}$
\[
{c_1}
\begin{bmatrix}
3 \\ 1 
\end{bmatrix}
{c_2}
\begin{bmatrix}
1 \\ 2
\end{bmatrix}
=
\begin{bmatrix}
107 \\ 64
\end{bmatrix}
\]
\[{c_1}3+{c_2}1=107\]
\[{c_1}1+{c_2}1=64\]
So, we can deduce with a little algebra that ${c_1}=30$ and ${c_2}=17$ and that
\[
30
\begin{bmatrix}
3 \\ 1 
\end{bmatrix}
17
\begin{bmatrix}
1 \\ 2
\end{bmatrix}
=
\begin{bmatrix}
107 \\ 64
\end{bmatrix}
\]
\section{Conclusion}
We have shown that using these two methods, that of the magic carpet, and that of
the hoverboard, that we can indeed move 30 times by hoverboard and 17 times by
magic carpet. Having done so, we will arive at old man Gauss's place
\end{document}